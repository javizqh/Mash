\documentclass[12pt,a4paper]{report}
\usepackage{graphicx} %LaTeX package to import graphics
\usepackage{enumitem}
\usepackage{geometry}
\usepackage{amssymb}
\usepackage[table]{xcolor}
\setlength{\arrayrulewidth}{0.5mm}
\setlength{\tabcolsep}{3pt}
\renewcommand{\arraystretch}{1.045}
\usepackage[utf8]{inputenc}
\usepackage{float}
\usepackage{lmodern}
\usepackage[T1]{fontenc}
\usepackage{hyperref}
\hypersetup{colorlinks=true, linkcolor=black, filecolor=magenta, urlcolor=cyan,}
\urlstyle{same}
\usepackage{amsmath}
\geometry{a4paper ,rmargin=10mm, lmargin=10mm}
\usepackage[export]{adjustbox}
\usepackage{titlesec}
\titleformat{\chapter}{\normalfont\huge}{\thechapter}{20pt}{\huge\bf}
\graphicspath{{media/}} %configuring the graphicx package
\title{mash documentation guide}
\author{Javier Izquierdo Hernández}
\date{\today}
\begin{document}
	\begin{titlepage}
		\centering
		{\includegraphics[width=0.3\textwidth]{logo}\par}
		\vspace{1cm}
		{\bfseries\LARGE Universidad Rey Juan Carlos \par}
		\vspace{1cm}
		{\scshape\Large Ingeniería de robótica software \par}
		\vspace{3cm}
		{\scshape\Huge Sistemas Operativos \par}
		\vspace{3cm}
		{\itshape\Large MASH Documentation Guide \par}
		\vfill
		{\Large Author: \par}
		{\Large Javier Izquierdo Hernández \par}
		\vfill
		{\Large \today \par}
	\end{titlepage}
	\newpage
	\tableofcontents
	\newpage
\part{Introduction}
Introduction on the mash shell
\part{Shell Grammar}
The following two tables form the syntax in the shell, being the first one the basic and the next one the extended. Next is some important information about it:\\
Only for ASCII characters, if used for another type it won't work as expected.\\
----- = Default or copy\\
Color = Change in syntax table\\
\textcolor[HTML]{C500FF}{Subexec} = Substitute by output of command inside
\begin{table}[H]
	\centering
	\begin{tabular}{ |c|c|c|c| }
		\hline
		\rowcolor{lightgray} \multicolumn{4}{|c|}{Basic Syntax Table} \\
		\hline
		Character & Standard & File \cellcolor[HTML]{BEE9F9}& Substitution \cellcolor[HTML]{FFF49C}\\
		\hline
		$\backslash$0 & end\_line & end\_file \cellcolor[HTML]{BEE9F9}&  end\_sub \cellcolor[HTML]{FFF49C}\\
		$\backslash$t & blank & end\_file\_started \cellcolor[HTML]{BEE9F9}&  end\_sub \cellcolor[HTML]{FFF49C}\\
		$\backslash$n & blank & end\_file \cellcolor[HTML]{BEE9F9}&  end\_sub \cellcolor[HTML]{FFF49C}\\
		Space & blank & end\_file\_started \cellcolor[HTML]{BEE9F9}&  end\_sub\cellcolor[HTML]{FFF49C}\\
		" & ----- & ----- &  end\_sub\cellcolor[HTML]{FFF49C}\\
		\# & ----- & end\_file \cellcolor[HTML]{BEE9F9}& copy\_and\_end\_sub \cellcolor[HTML]{FFF49C}\\
		\$ & start\_sub \cellcolor[HTML]{FFF49C}& start\_sub \cellcolor[HTML]{FFF49C} & copy\_and\_end\_sub \cellcolor[HTML]{FFF49C}\\
		\& & background & end\_file \cellcolor[HTML]{BEE9F9}&  end\_sub\cellcolor[HTML]{FFF49C}\\
		' & ----- & ----- &  end\_sub\cellcolor[HTML]{FFF49C}\\
		$($ & error \cellcolor[HTML]{FF0044}& error \cellcolor[HTML]{FF0044} & end\_sub \cellcolor[HTML]{FFF49C}\\
		$)$ & error \cellcolor[HTML]{FF0044}& error \cellcolor[HTML]{FF0044} & end\_sub \cellcolor[HTML]{FFF49C}\\
		* & do\_glob & do\_glob & end\_sub\cellcolor[HTML]{FFF49C} \\
		- & ----- & ----- &  copy\_and\_end\_sub \cellcolor[HTML]{FFF49C}\\
		; & ----- & end\_file\cellcolor[HTML]{BEE9F9} &  end\_sub \cellcolor[HTML]{FFF49C}\\
		< & start\_file\_in \cellcolor[HTML]{BEE9F9}& end\_file\cellcolor[HTML]{BEE9F9}&  end\_sub \cellcolor[HTML]{FFF49C}\\
		> & start\_file\_out \cellcolor[HTML]{BEE9F9}& end\_file \cellcolor[HTML]{BEE9F9}&  end\_sub \cellcolor[HTML]{FFF49C}\\
		? & do\_glob & do\_glob & copy\_and\_end\_sub \cellcolor[HTML]{FFF49C}\\
		@ & ----- & ----- & copy\_and\_end\_sub \cellcolor[HTML]{FFF49C}\\
		$[$ & do\_glob & do\_glob &  end\_sub \cellcolor[HTML]{FFF49C}\\
		$\backslash$ & ----- & ----- &  end\_sub \cellcolor[HTML]{FFF49C}\\
		\_ & ----- & end\_file \cellcolor[HTML]{BEE9F9}& copy\_and\_end\_sub \cellcolor[HTML]{FFF49C}\\
		\{ & here\_doc & error \cellcolor[HTML]{FF0044} &  end\_sub \cellcolor[HTML]{FFF49C}\\
		\} & error \cellcolor[HTML]{FF0044} & error \cellcolor[HTML]{FF0044} &  end\_sub \cellcolor[HTML]{FFF49C} \\
		$|$ & pipe\_tok & end\_file\cellcolor[HTML]{BEE9F9} &  end\_sub \cellcolor[HTML]{FFF49C}\\
		\textasciitilde  & ----- & ----- &  end\_sub \cellcolor[HTML]{FFF49C}\\
		Rest  & ----- & ----- &  ----- \\
		\hline
	\end{tabular}
\end{table}
\begin{table}[H]
\begin{tabular}{ |c|c|c|c|c|c| }
	\hline
	\rowcolor{lightgray} \multicolumn{6}{|c|}{Extended Syntax Table} \\
	\hline
	Character & Standard & File \cellcolor[HTML]{BEE9F9}& Substitution \cellcolor[HTML]{FFF49C}& 'Single Quote' \cellcolor[HTML]{E1E1E1} & "Double Quote"\cellcolor[HTML]{C8F3BE} \\
	\hline
	$\backslash$0 & end\_line & end\_file \cellcolor[HTML]{BEE9F9}&  end\_sub \cellcolor[HTML]{FFF49C}& request\_new\_line & request\_new\_line \\
	$\backslash$t & blank & end\_file\_started \cellcolor[HTML]{BEE9F9}&  end\_sub \cellcolor[HTML]{FFF49C}& ----- & ----- \\
	$\backslash$n & blank & end\_file \cellcolor[HTML]{BEE9F9}&  end\_sub \cellcolor[HTML]{FFF49C}& ----- & request\_new\_line \\
	Space & blank & end\_file\_started \cellcolor[HTML]{BEE9F9}&  end\_sub\cellcolor[HTML]{FFF49C} & ----- & ----- \\
	" & start\_dquote \cellcolor[HTML]{C8F3BE}& start\_dquote \cellcolor[HTML]{C8F3BE}&  end\_sub\cellcolor[HTML]{FFF49C} & ----- & end\_dquote \cellcolor[HTML]{C8F3BE}\\
	\# & comment & end\_file \cellcolor[HTML]{BEE9F9}& copy\_and\_end\_sub \cellcolor[HTML]{FFF49C}& ----- & ----- \\
	\$ & start\_sub \cellcolor[HTML]{FFF49C}& start\_sub \cellcolor[HTML]{FFF49C} & copy\_and\_end\_sub \cellcolor[HTML]{FFF49C}& ----- & start\_sub \cellcolor[HTML]{FFF49C} \\
	\& & background & end\_file \cellcolor[HTML]{BEE9F9}&  end\_sub\cellcolor[HTML]{FFF49C} & ----- & ----- \\
	' & start\_squote \cellcolor[HTML]{E1E1E1}& start\_squote \cellcolor[HTML]{E1E1E1}&  end\_sub\cellcolor[HTML]{FFF49C} & end\_squote \cellcolor[HTML]{E1E1E1} & ----- \\
	$($ & error \cellcolor[HTML]{FF0044}& error \cellcolor[HTML]{FF0044} & end\_sub \cellcolor[HTML]{FFF49C}& ----- & ----- \\
	$)$ & error \cellcolor[HTML]{FF0044}& error \cellcolor[HTML]{FF0044} & end\_sub \cellcolor[HTML]{FFF49C}& ----- & ----- \\
	* & do\_glob & do\_glob & end\_sub\cellcolor[HTML]{FFF49C} & ----- & ----- \\
	- & ----- & ----- &  copy\_and\_end\_sub \cellcolor[HTML]{FFF49C}& ----- & ----- \\
	; & end\_pipe & end\_file\cellcolor[HTML]{BEE9F9} &  end\_sub \cellcolor[HTML]{FFF49C}& ----- & ----- \\
	< & start\_file\_in \cellcolor[HTML]{BEE9F9}& end\_file\cellcolor[HTML]{BEE9F9}&  end\_sub \cellcolor[HTML]{FFF49C}& ----- & ----- \\
	> & start\_file\_out \cellcolor[HTML]{BEE9F9}& end\_file \cellcolor[HTML]{BEE9F9}&  end\_sub \cellcolor[HTML]{FFF49C}& ----- & ----- \\
	? & do\_glob & do\_glob & copy\_and\_end\_sub \cellcolor[HTML]{FFF49C}& ----- & ----- \\
	@ & ----- & ----- & copy\_and\_end\_sub \cellcolor[HTML]{FFF49C}& ----- & ----- \\
	$[$ & do\_glob & do\_glob &  end\_sub \cellcolor[HTML]{FFF49C}& ----- & ----- \\
	$\backslash$ & escape & escape &  end\_sub \cellcolor[HTML]{FFF49C}& ----- & esp\_escape \\
	\_ & ----- & end\_file \cellcolor[HTML]{BEE9F9}& copy\_and\_end\_sub \cellcolor[HTML]{FFF49C}& ----- & ----- \\
	\{ & here\_doc & error \cellcolor[HTML]{FF0044} &  end\_sub \cellcolor[HTML]{FFF49C}& ----- & ----- \\
	\} & error \cellcolor[HTML]{FF0044} & error \cellcolor[HTML]{FF0044} & end\_sub \cellcolor[HTML]{FFF49C} & ----- & ----- \\
	$|$ & pipe\_tok & end\_file\cellcolor[HTML]{BEE9F9} &  end\_sub \cellcolor[HTML]{FFF49C}& ----- & ----- \\
	\textasciitilde  & tilde\_tok \cellcolor[HTML]{FFF49C}& tilde\_tok \cellcolor[HTML]{FFF49C} &  end\_sub \cellcolor[HTML]{FFF49C}& ----- & ----- \\
	Rest  & ----- & ----- &  ----- & ----- & ----- \\
	\hline
	\&\&  & and & end\_file \cellcolor[HTML]{BEE9F9}&  end\_sub\cellcolor[HTML]{FFF49C} & ----- & ----- \\
	$||$  & or & end\_file\cellcolor[HTML]{BEE9F9} &  end\_sub \cellcolor[HTML]{FFF49C}& ----- & ----- \\
	\$$($  & subexec\cellcolor[HTML]{C500FF} & subexec\cellcolor[HTML]{C500FF} &  ----- & ----- & subexec\cellcolor[HTML]{C500FF} \\
	\&>  & start\_file\_out \cellcolor[HTML]{BEE9F9}& end\_file \cellcolor[HTML]{BEE9F9}&  end\_sub \cellcolor[HTML]{FFF49C}& ----- & ----- \\
	2>  & start\_file\_out \cellcolor[HTML]{BEE9F9}& end\_file \cellcolor[HTML]{BEE9F9}&  end\_sub \cellcolor[HTML]{FFF49C}& ----- & ----- \\
	1>  & start\_file\_out \cellcolor[HTML]{BEE9F9}& end\_file \cellcolor[HTML]{BEE9F9}&  end\_sub \cellcolor[HTML]{FFF49C}& ----- & ----- \\
	\hline
\end{tabular}
\end{table}
\part{Redirections}
\setcounter{chapter}{1}
\section*{Redirecting input}
Redirection of input causes the file to be opened for reading on the standard input.\\
\newline
The general format for redirecting input is:\\
\newline
\hspace*{7 mm}<filename
\section*{Redirecting Standard Output}
Redirection of output causes the file to be opened for reading on the standard output. If the file does not exist it is created; if it does exist it is truncated to zero size.\\
\newline
The general format for redirecting output is:\\
\newline
\hspace*{7 mm}>filename
\section*{Redirecting Standard Error}
Redirection of error causes the file to be opened for reading on the standard error. If the file does not exist it is created; if it does exist it is truncated to zero size.\\
\newline
The general format for redirecting error is:\\
\newline
\hspace*{7 mm}2>filename
\section*{Redirecting Standard Output and Standard Error}
This  construct  allows both the standard output (file descriptor 1) and the standard error output (file descriptor 2) to be redirected to the file.\\
\newline
The general format for redirecting output and error is:\\
\newline
\hspace*{7 mm}\&>filename
\section*{Here documents}
This type of redirection instructs the shell to read input from the current source until a line containing only delimiter '\}' (with no trailing characters) is  seen.   All  of  the
lines read up to that point are then used as the standard input for a command.\\
\newline
The general format for here documents is:\\
\newline
\hspace*{7 mm}HERE\{\\
\hspace*{14 mm} here-document\\
\hspace*{7 mm} \}\\

\part{Prompting}
\setcounter{chapter}{0}
When  executing  interactively, mash displays the primary prompt PROMPT when it is ready to read a command, and '> ' when it needs more input to complete a command.\\
Mash allows these prompt strings to be customized by inserting a number of special strings that are decoded as follows:
\begin{itemize}
	\item @-\hspace{7 mm}does nothing
	\item @ifcustom\hspace{7 mm}if file named .mash\_prompt is found in directory the enter if statement
	\item @ifgit\hspace{7 mm}if directory is a git repository then enter if statement
	\item @else\hspace{7 mm}if not enter if statement then exec the following
	\item @endif\hspace{7 mm}end of if statement
	\item @user\hspace{7 mm}the username of the current user
	\item @where\hspace{7 mm}the current working directory, with \$HOME abbreviated with a tilde
	\item @host\hspace{7 mm}the hostname
	\item @custom\hspace{7 mm}replace with contents of file .mash\_prompt. Use with @ifcustom.
	\item @gitstatuscolor\hspace{7 mm}if git is up to date show color green, if not red
	\item @gitstatus\hspace{7 mm}the number of files to add to commit
	\item @branch\hspace{7 mm}the name of the git branch
	\item @black\hspace{7 mm}show color black
	\item @red\hspace{7 mm}show color red
	\item @green\hspace{7 mm}show color green
	\item @yellow\hspace{7 mm}show color yellow
	\item @blue\hspace{7 mm}show color blue
	\item @pink\hspace{7 mm}show color pink
	\item @cyan\hspace{7 mm}show color cyan
	\item @white\hspace{7 mm}show color white
	\item @nocolor\hspace{7 mm}remove all colors
\end{itemize}
\part{Builtins}
\setcounter{chapter}{0}
\chapter{Arithmetic evaluation}
\section{Description}
Write result of the arithmetic expression to the standard output.
\section{Usage}
\$(( expression ))
\section{Extended description}
Evaluation is done in fixed-width integers with no check for overflow, though division by 0 is considered as an error.  The operators and  their  precedence,  associativity,  and values are the same as in the C language.\\
Shell variables are allowed as operands; parameter expansion is performed before the expression is evaluated.  Within an expression, shell variables may also be referenced by name without using the \$ character.
\section{Operands}
\begin{itemize}
	\item +\hspace{7 mm}addition
	\item -\hspace{7 mm}subtraction	
	\item *\hspace{7 mm}multiplication
	\item /\hspace{7 mm}division
	\item \hspace{2 mm}$\widehat{ }$\hspace{7 mm}exponentiation
\end{itemize}
\section{Exit Status}
Returns success unless an error in the expression is found.
\newpage

\chapter{Alias}
\section{Description}
Define or display aliases.
\section{Usage}
alias [name=value]
\section{Extended description}
Without arguments, `alias' prints the list of aliases in the reusable
form `alias NAME=VALUE' on standard output.\\

\noindent Otherwise, an alias is defined for each NAME whose VALUE is given.
\section{Exit Status}
alias returns 0 unless a VALUE is missing or the is out of memory.
\newpage
\chapter{Bg}
\section{Description}
Move jobs to the background.
\section{Usage}
bg [jobspec]
\section{Extended description}
Place the jobs identified by the JOB$\_$SPEC in the background, as if they had been started with `\&'. \\

\noindent If JOB$\_$SPEC is not present, the shell's notion of the current job is used.
\section{Exit Status}
Returns success unless job control is not enabled or an error occurs.
\newpage

\chapter{Builtin}
\section{Description}
Execute shell builtins.
\section{Usage}
builtin shell-builtin [arg ..]
\section{Extended description}
Execute SHELL-BUILTIN with arguments ARGs without performing command lookup.
\section{Exit Status}
Returns the exit status of SHELL-BUILTIN, or 1 if SHELL-BUILTIN not a shell builtin.
\newpage

\chapter{Cd}
\section{Description}
Change the shell working directory.
\section{Usage}
cd [directory]
\section{Extended description}
Change the current directory to DIR.  The default DIR is the value of the HOME shell variable.
\section{Exit Status}
Returns 0 if the directory is changed, and non-zero otherwise.
\newpage

\chapter{Command}
\section{Description}
Execute a simple command or display information about commands.
\section{Usage}
command [-Vv] command [arg ..]
\section{Extended description}
Runs COMMAND with ARGS suppressing  shell function lookup, or display information about the specified COMMANDs.
\section{Options}
\begin{itemize}
	\item -v\hspace{7 mm}print a description of COMMAND similar to the `type' builtin
	\item -V\hspace{7 mm}print a more verbose description of each COMMAND
\end{itemize}
\section{Exit Status}
Returns exit status of COMMAND, or failure if COMMAND is not found.
\newpage

\chapter{Disown}
\section{Description}
Remove jobs from current shell.
\section{Usage}
disown [-ar] [jobspec … | pid … ]
\section{Extended description}
Removes each JOBSPEC argument from the table of active jobs. Without any JOBSPECs, the shell uses its notion of the current job.
\section{Options}
\begin{itemize}
	\item -a\hspace{7 mm}remove all jobs if JOBSPEC is not supplied
	\item -r\hspace{7 mm}remove only running jobs
\end{itemize}
\section{Exit Status}
Returns success unless an invalid option or JOBSPEC is given, or job control is not enabled.
\newpage

\chapter{Echo}
\section{Description}
Write arguments to the standard output.
\section{Usage}
echo [-n] [arg ...]
\section{Extended description}
Display the ARGs, separated by a single space character and followed by a newline, on the standard output.
\section{Options}
\begin{itemize}
	\item -n\hspace{7 mm}do not append a newline
\end{itemize}
\section{Exit Status}
Returns success unless a write error occurs.
\newpage

\chapter{Exit}
\section{Description}
Exit the shell.
\section{Usage}
exit [n]
\section{Extended description}
Exits the shell with a status of N.  If N is omitted, the exit status is that of the last command executed.
\newpage

\chapter{Export}
\section{Description}
Set export attribute for shell variables.
\section{Usage}
export [name=value]
\section{Extended description}
Marks each NAME for automatic export to the environment of subsequently executed commands.  If VALUE is supplied, assign VALUE before exporting.
\section{Exit Status}
Returns success unless an invalid option is given or NAME is invalid.
\newpage

\chapter{Fg}
\section{Description}
Move job to the foreground.
\section{Usage}
fg [jobspec]
\section{Extended description}
Place the job identified by JOB$\_$SPEC in the foreground, making it the current job.  If JOB$\_$SPEC is not present, the shell's notion of the current job is used.
\section{Exit Status}
Status of command placed in foreground, or failure if an error occurs or job control is not enabled.
\newpage

\chapter{Help}
\section{Description}
Display information about builtin commands.
\section{Usage}
help [-dms] [pattern ...]
\section{Extended description}
Displays brief summaries of builtin commands.  If PATTERN is specified, gives detailed help on all commands matching PATTERN, otherwise the list of help topics is printed.
\section{Options}
\begin{itemize}
	\item -d\hspace{7 mm}output short description for each topic
	\item -m\hspace{7 mm}display usage in pseudo-manpage format
	\item -s\hspace{7 mm}output only a short usage synopsis for each topic matching PATTERN
\end{itemize}
\section{Arguments}
\begin{itemize}
	\item PATTERN\hspace{7 mm}Pattern specifying a help topic
\end{itemize}
\section{Exit Status}
Returns success unless PATTERN is not found or an invalid option is given.
\newpage

\chapter{Ifnot}
\section{Description}
Execute the command if previous command failed.
\section{Usage}
ifnot command [arg ..]
\section{Extended description}
Runs COMMAND with ARGS if the last executed command finished with failure.
\section{Exit Status}
Returns exit status of COMMAND, or success if last command ended with success.\\
If enviroment variable 'result' does not exist returns failure.
\newpage

\chapter{Ifok}
\section{Description}
Execute the command if previous command ended successfully.
\section{Usage}
ifok command [arg ..]
\section{Extended description}
Runs COMMAND with ARGS if the last executed command finished with success.
\section{Exit Status}
Returns exit status of COMMAND, or success if last command ended with failure.\\
If enviroment variable 'result' does not exist returns failure.
\newpage

\chapter{Jobs}
\section{Description}
Display status of jobs.
\section{Usage}
jobs [-lprs] [jobspec]
\section{Extended description}
Lists the active jobs.  JOBSPEC restricts output to that job.\\
Without options, the status of all active jobs is displayed.
\section{Options}
\begin{itemize}
	\item -l\hspace{7 mm}lists process IDs in addition to the normal information
	\item -p\hspace{7 mm}lists process IDs only
	\item -r\hspace{7 mm}restrict output to running jobs
	\item -s\hspace{7 mm}restrict output to stopped jobs
\end{itemize}
\section{Exit Status}
Returns success unless an invalid option is given, an error occurs or job control is not enabled.
\newpage

\chapter{Kill}
\section{Description}
Send a signal to a job.
\section{Usage}
kill [-s sigspec] | [-n signum] | [-sigspec] jobspec or pid or kill -l [sigspec]
\section{Extended description}
Send the processes identified by PID or JOBSPEC the signal named by SIGSPEC or SIGNUM.\\
If neither SIGSPEC nor SIGNUM is present, then SIGTERM is assumed.
\section{Options}
\begin{itemize}
	\item -s sig\hspace{7 mm}SIG is a signal name
	\item -n sig\hspace{7 mm}SIG is a signal number
	\item -l\hspace{14 mm}list the signal names; if an argument follows `-l' it is assumed to be a signal name\\
	\hspace*{17 mm}for which number should be listed
	\item -L\hspace{14 mm}synonym for -l
\end{itemize}
\section{Exit Status}
Returns success unless an invalid option is given or an error occurs or job control is not enabled.
\newpage

\chapter{Math}
\section{Description}
Write result of the arithmetic expression to the standard output.
\section{Usage}
math expression
\section{Extended description}
Display the result of the arithmetic expression, followed by a newline, on the standard output.
\section{Operands}
\begin{itemize}
	\item +\hspace{7 mm}addition
	\item -\hspace{7 mm}subtraction	
	\item *\hspace{7 mm}multiplication
	\item /\hspace{7 mm}division
	\item \hspace{2 mm}$\widehat{ }$\hspace{7 mm}exponentiation
\end{itemize}
\section{Exit Status}
Returns success unless an error in the expression is found.
\newpage

\chapter{Pwd}
\section{Description}
Print the name of the current working directory.
\section{Usage}
pwd
\section{Extended description}
Print the name of the current working directory.
\section{Exit Status}
Returns 0 unless an invalid option is given or the current directory cannot be read.
\newpage

\chapter{Sleep}
\section{Description}
Pause for NUMBER seconds.
\section{Usage}
sleep NUMBER[SUFFIX]...
\section{Extended description}
Pause for NUMBER seconds.  SUFFIX may be 's' for seconds (default),'m' for minutes, 'h' for hours or 'd' for days.\\
NUMBER need to be an integer.\\
Given two or more arguments, pause for the amount of time specified by the sum of their values.
\section{Exit Status}
Returns success unless an invalid option or time is given.
\newpage

\chapter{Source}
\section{Description}
Execute commands from a file in the current shell.
\section{Usage}
source filename
\section{Extended description}
Read and execute commands from FILENAME in the current shell.  The entries in \$PATH are used to find the directory containing FILENAME.
\section{Exit Status}
Returns success unless FILENAME cannot be read.
\newpage

\chapter{Wait}
\section{Description}
Wait for job completion and return exit status.
\section{Usage}
wait [jobspec or id]
\section{Extended description}
Waits for each process identified by an ID, which may be a process ID or a job specification, and reports its termination status.\\
If ID is not given, waits for the current active child processes, and the return it's status. If ID is a job specification, waits for all processes in that job's pipeline.
\section{Exit Status}
Returns the status of the last ID; fails if ID is invalid, stopped, an invalid option is given or job control is not enabled.
\newpage
\begin{thebibliography}{100} % 100 is a random guess of the total number of
	%references
	\bibitem{Bash} Brian Fox and Chet Ramey, ``Bash man page``.
	\bibitem{Rc} Byron Rakitzis, ``Rc shell``.
\end{thebibliography}
\end{document}